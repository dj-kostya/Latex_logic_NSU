\documentclass{article}
\usepackage[utf8]{inputenc}
\usepackage[english, russian]{babel}
\usepackage{indentfirst}
\usepackage{amsmath,amsfonts,amssymb,amsthm,mathtools}
\usepackage{svg}
\newtheorem{theorem}{Теорема}
\newtheorem*{theorem*}{Теорема}
\newtheorem{lemma}{Лемма}
\newtheorem*{lemma*}{Лемма}
\theoremstyle{definition}
\newtheorem{definition}{Определение}
\newtheorem*{definition*}{Определение}
\newtheorem*{name}{Обозначение}
\newtheorem*{exmp}{Пример}
\newtheorem*{paradoks}{Парадокс}
\newtheorem*{hypo}{Гипотеза}
\newtheorem{prorosition}{Предложение}
\newtheorem*{proposition*}{Предложение}
\title{ Коллоквиум №1 (20.11.2019) }
\author{GROUPS №19137,№19144}
\date{2019\\}
\begin{document}
\maketitle
\begin{enumerate}
 \item Множество: способы задания, операции над множествами
       \\ Не существует явного определения множества.
       \\ Пусть A некоторое мн-во, тогда существует 2 способа задания мн-ва
       \begin{enumerate}
        \item A = \{1,2,3,4,5\} - явное задание эл-тов мн-ва \\
        \item Пусть $\Phi(x)$- некоторое условие, тогда \\A = $\{x \ | \ \Phi(x) \}$ - Задание множествами с помощью некоторого условия $\Phi(x)$
       \end{enumerate}
       Пусть A, B- некоторые множества \\
       \begin{name}[Подмножетсво]
        A - подмножетсво B, если
        \mbox{$A \subseteq B = \{x \ | x\in{A} \Rightarrow x\in{B} \}$}
       \end{name}
       \begin{name}[Собстевенное подмножетсво]
        A - собстевенное подмножетсво B, если
        \mbox{$A \subset B$, если $A \subseteq B$ и $A\ne{B} $}
       \end{name}
       \begin{name}[Пустое множество]
        $\emptyset$ - множество, не содержащее эл-тов ("Пустое множество")
       \end{name}
       \begin{name}[Множество всех подмножетсв множества A]
        \mbox{$P(A) = \{ \ C\ |\ C \subseteq{A} \ \} $}
       \end{name}
       \begin{name}[Универсум]
        Универсум (условное множество все множеств) $U$
       \end{name}
       Операции над множествами:
       \begin{itemize}
        \item Объединение множеств:
              \\ $A\cup{B} = \{ x \ | \ x \in{A} \lor x\in{B}\}$
        \item Пересечение множеств:
              \\ $A\cap{B} = \{ x \ | \ x \in{A} \land x\in{B}\}$
        \item Разность множеств:
              \\ $A \ \backslash\  B = \{x\ | \ x\in{A} \land x\notin{B} \}$
        \item Дополнение множества:
              \\ $\neg{A} = \{\ x\ |\ x \in{\ U} \land{x \notin{A}}\} $
        \item Симметрическая разность множеств:
              \\ $A \ \Delta \ B = (A \  \backslash \ B) \cup{(B \ \backslash{\ A})} = (A\ \cup{\ B })\ \backslash \ (B\ \cup{\ A } ) $
       \end{itemize}
       \newpage
       Пусть S - семейство множеств:
       \begin{itemize}
        \item Объединение семейства множеств\\
              $\bigcup{S} = \{\ x\ |\ \exists{A_i} \in{\ S \ }:x\ \in{\ A_i \ }\}$
        \item Пересечение семейства множеств\\
              $\bigcap{S} = \{\ x\ |\ \forall{A_i} \in{\ S\ }: x\ \in{\ A_i \ } \}$
       \end{itemize}
 \item Упорядоченный набор (кортеж), предложение о равенстве n-ок, декартово произведение, декартова степень.
       \begin{definition*}[Упорядоченный набор (кортеж)]
        Упорядоченный набор (кортеж) длинны n определяется по индукции
        \begin{math}\\
         <> = \emptyset\\
         <a> = a\\
         <a,b> = \{\{a\},\{a,b\}\}
         \\
         ...
         \\
         <a_1, a_2, ...,a_{n-1} ,a_n> = <<a_1, a_2, ...,a_{n-1})>, a_n>
        \end{math}
       \end{definition*}
       \begin{definition*}[пара]
        Набор $<a,b>$ длинны 2 называют \textit{парой}
       \end{definition*}
       \begin{proposition*}[о равенстве n-ок]
        Если \\ \mbox{$<a_1,...,a_n> = <b_1, ..., b_n> \Leftrightarrow a_1 = b_1, ..., a_n = b_n$}
       \end{proposition*}
       \begin{proof}
        для $n=1$ очевидно в обе стороны. Докажем для $n = 2$: \\
        $<a_1, a_2> = <b_1, b_2> \Leftrightarrow \{\{a_1\},\{ a_1, a_2\} \} = \{\{b_1\},\{ b_1, b_2\}\}$
        \\ Пусть $a_1 = a_2 \Rightarrow \left[
         \begin{gathered}
          \{a_1\} = \{b_1, b_2\}\hfill \\
          \{a_1, a_2\} = \{b_1, b_2\}\hfill \\
         \end{gathered}
         \right. \Rightarrow a_1=a_2 = b_1 = b_2 $
        \\ для $b_1 = b_2$ аналогично.\\ Расмотрим $a_1 \ne a_2, b_1 \ne b_2$\\
        $\Rightarrow \left[ \begin{gathered}
          \{a_1\} = \{b_1\}\hfill \\
          \{a_1\} = \{b_1, b_2\}\hfill \\
         \end{gathered}
         \right. \Rightarrow \{a_1\} = \{b_1\} \Rightarrow a_1 = b_1$\\
        \\По аналогии для $\{a_1, a_2\} = \{b_1, b_2\}$\\
        Т.к справледливо для $n=2$, а определение n-ок индуктивно следовательно верно для n
       \end{proof}
       \begin{definition*}[Декартово произведение]
        Пусть даны множества $A_1,...,A_n$, тогда их декартовым произведением называют
        \newline\mbox{$A_1 \times A_2 \times ... \times A_n = \{<a_1,...,a_n>|\  \forall{i}\in{\{1,...,n\}}\ a_i\in{A_i} \}$}
       \end{definition*}
       \begin{definition*}[Декартова степень]
        В случае, если \mbox{$A_1=A_2=...=A_n$}, тогда $A_1 \times A_2 \times ... \times A_n$ называют декартовой степенью и обозначают, как $A^n=A_1 \times A_2 \times ... \times A_n$
       \end{definition*}
       \newpage
 \item Бинарные отношения, обратное отношение, произведение отношений, лемма о бинарных отношениях.
       \begin{definition*}Бинарным отношением между элементами множеств A и B называется произвольное подмножество\\
        $C \subseteq{A \times B}$
       \end{definition*}
       \begin{definition*}Обратным бинарным отношением называется \\
        $R^{-1} = \{<y;x> |\ <x;y> \in{R}\}$
       \end{definition*}
       \begin{definition*}Произведением бинарных отношений называется \\
        $R_{1}\times R_{2} = \{<x;z>          |\exists z |\ <x;y> \in R_{1}   \land <y;z> \in R_{2}\}$
       \end{definition*}
       \begin{lemma*}[Лемма о бинарных отношениях]
        Для любых бинарных отношений $R_{1},R_{2},R_{3} $:
        \begin{enumerate}
         \item     $R_{1}\cdot (R_{2}\cdot R_{3}) = (R_{1}\cdot R_{2})\cdot R_{3} $\\
         \item $(R_{1}\cdot R_{2})^{-1} = R_{2}^{-1}\cdot R_{1}^{-1} $
        \end{enumerate}
       \end{lemma*}
       \begin{proof}
        \begin{enumerate}
         \item Покажем, что $R_{1}\cdot (R_{2}\cdot R_{3}) \subseteq (R_{1}\cdot R_{2})\cdot R_{3} $. \\
               Пусть $<x;t> \in R_{1}\cdot (R_{2}\cdot R_{3})$, тогда существует $y$ такое, что \\ $<x;y>\in R_{1}$ и $<y;t>\in R_{2}\cdot R_{3}$. Далее существует $z$ такое, что  $<y;z>\in R_{2}$ и  $<z;t>\in R_{3}$. Получаем, что $<x;z>\in R_{1}\cdot R_{2}$ и $<x;t>\in (R_{1}\cdot R_{2})\cdot R_{3}$.
               Обратное включение доказывается аналогично.
         \item Покажем, что $(R_{1}\cdot R_{2})^{-1} \subseteq R_{2}^{-1}\cdot R_{1}^{-1} $.\\
               Пусть $<z;x> \in (R_{1}\cdot R_{2})^{-1}$, тогда существует $y$ такое, что $<x;y>\in R_{1}$ и $<y;z>\in R_{2}$. Тогда $<y;x>\in R_{1}^{-1}$ и
               $<z;y>\in R_{2}^{-1}$.
               Получаем, что $<z;x> \in R_{2}^{-1}\cdot R_{1}^{-1} $.
               Обратное включение доказывается аналогично.
        \end{enumerate}
       \end{proof}
 \item Композиция функций, лемма о композиции функций:
       \begin{definition*}[Композиция функций]
        Если $f$ и $g$ - функции, то их \textit{композиция} $g\circ{f}$ определяется, как произведение бинарных отношений $f\cdot{g}$ (В обратном порядке)
       \end{definition*}
       \begin{lemma*}[о композиции функций]
        Если $f: A \rightarrow B, g:B\rightarrow C$, то их композицией $g\circ{f} : A\rightarrow C$
        и $[g\circ{f}](x)=g(f(x))$ при $x\in{A}$
       \end{lemma*}
 \item Сюръекция, инъекция, биекция, обратная функция, лемма о свойствах биекций
       \\ Пусть $ f: A \rightarrow B$
       \begin{definition*}[Сюръекция]
        $f$ - функция из A \textit{на} B (\textit{сюръективная функция, сюръекция}), если $\forall y \in B$ $\exists x \in A$ | $f(x) = y$
       \end{definition*}
       \begin{name}[Сюръекция]
        $f: A\xrightarrow[\textit{на}]{}B$.
       \end{name}
       \begin{definition*}[Инъекция]
        $f$ - инъективная функция (\textit{1 - 1 функция, инъекция}), если $\forall x_{1}, x_{2}\in A $ из $f(x_{1}) = f(x_{2})$ следует $x_{1} = x_{2}$
       \end{definition*}
       \begin{name}[Инъекция]
        $f: A \xrightarrow{1-1} B$
       \end{name}
       \begin{definition*}[Биекция]
        $f$ - \textit{биекция} из A на B, если $f$ одновременно и инъекция, и сюръекция.
       \end{definition*}
       \begin{name}[Биекция]
        $f: A \xrightarrow[\textit{на}]{1-1} B$
       \end{name}
       \begin{definition*}[Обратная функция]
        Запись $f^{-1}$ означает обратное бинарное отношение к $f$. Если $f^{-1}$ при этом является функцией, то она называется \textit{обратной функцией} к $f$.
       \end{definition*}

       \begin{lemma*}[о свойствах биекций]\mbox{}\\
        \begin{enumerate}
         \item Если $f: A \xrightarrow[\textit{на}]{1-1} B$, то $f^{-1}: B \xrightarrow[\textit{на}]{1-1} A$, $f^{-1}(f(x)) = x$ $\forall x \in A$ и $f(f^{-1}(y)) = y$ $\forall y \in B$.
         \item Если $f: A \xrightarrow[\textit{на}]{1-1} B$, $g: B \xrightarrow[\textit{на}]{1-1} C$, то $f \circ g:  A \xrightarrow[\textit{на}]{1-1} C$.
        \end{enumerate}
       \end{lemma*}


       \begin{proof}
        \begin{enumerate}
         \item Покажем, что $f^{-1}$ - функция.\\ Пусть \mbox{$<y, x_1>, <y, x_2> \in f^{-1}$}. Тогда $<x_1, y>, <x_2, y>\in f$ и $f(x_1) = f(x_2) = y$. Поскольку $f$ инъективна, $x_1 = x_2$.\\
               Ясно, что $dom(f^{-1}) = ran(f)$ и $ran(f^{-1}) = dom(f)$. Поскольку $f$ сюръективна, $ran(f) = B = dom(f^{-1})$. Поскольку $ran(f^{-1}) = A$, $f^{-1}$ сюръективна. Инъективность $f^{-1}$ легко проверяется. Тем самым $f^{-1}: B\xrightarrow[\textit{на}]{1-1} A$.\\
               Покажем, что $f^{-1}(f(x)) = x$ при $x \in A$. Пусть  $x \in A$ и $y = f(x)$. Тогда $<x, y>\in f$ и $<y, x> \in f^{-1}$. Получаем, что $f^{-1}(y) = x$.
         \item выше доказано, что $g \circ f: A \rightarrow C$ и $[g \circ f](x) = g(f(x))$. Инъективность: если $g(f(x_1)) = g(f(x_2))$, то $f(x_1) = f(x_2)$ и отсюда $x_1 = x_2$. Сюръективность доказывается похожим способом.
        \end{enumerate}

       \end{proof}
 \item Полный порядок, в.у.м., лемма о начальных сегментах в.у.м.
       \begin{definition*}[Вполне упорядоченное множество]
        \textit{Вполне упорядоченное множество} (в.у.м) - это пара $(A, \leq)$, где $\leq$ - линейный фундированный порядок на $A$. Иногда такой порядок называют \textit{полным}.
       \end{definition*}

       \begin{lemma*}[о начальных сегментах в.у.м.]
        Любой начальный сегмент в.у.м. $(A, \leq)$ либо равен $A$, либо является начальным отрезком.
       \end{lemma*}
       \begin{proof}
        Пусть $S$ - начальный сегмент в $A$ и $S \neq A$. Тогда $A \textbackslash S \neq \emptyset$. Пусть $x$ - минимальный элемент в $A \textbackslash S$. Покажем, что $S = A_x$. Если $y \in S$, то либо $y < x$, либо $x \leq y$. Второй случай невозможен, так как тогда $x \in S$.
       \end{proof}
 \item Парадокс Рассела, аксиоматика ZFC.
       \begin{paradoks}[Парадокс Рассела]
        Рассмотрим совокупность:
        $M_{R} = \{ A \mid  A$ - множество и $A  \notin A \}$.\\
        Предположим, что само $M_{R}$ является множеством. Возможны два варианта:\\
        \begin{enumerate}
         \item $M_{R} \notin{M_{R}}$. Тогда $A - M_{R}$ подходит под определние, и $M_{R} \notin{M_{R}}$. Противоречие.
         \item $M_{R} \in{M_{R}}$. Вновь полагая, $A = M_{R}$, получаем, что по определению $M_{R} \notin{M_{R}}$. Противоречие.
        \end{enumerate}
        Это рассуждение показывает, что совокупность $M_{R}$ нельзя считать множеством.
       \end{paradoks}
       Аксиоматика ZFC.\\
       Можно с собой на листочке!!!
 \item Лемма о порядке на мощностях.
       \begin{lemma*}[Лемма о порядке на мощностях]
        Для всяких непустых множеств $A$ и $B$ следующие условия эквиваленты:
        \begin{enumerate}
         \item $\left | A \right | \leq \left | B \right |$
         \item Существует функция $g : B\xrightarrow[]{HA}A$
         \item $A$ равномощно некоторому подмножеству $B$
        \end{enumerate}
       \end{lemma*}
       \begin{proof}
        \mbox{}\\
        \begin{enumerate}
         \item $a \Rightarrow c$ \\
               Пусть $\left | A \right | \leq \left | B \right |$.\\
               Тогда существует $f : A \xrightarrow[]{1-1} B$. \\
               Тогда $ran(f) \subseteq B$ и $f : A\xrightarrow[\textit{на}]{1-1}ran(f)$.\\
         \item $c \Rightarrow b$ \\
               Пусть $h : B_{1}\xrightarrow[\text{на}]{1-1}A$, где $B_{1}\subseteq B$. \\
               Выберем произвольное $a_{0} \in A$ и построим $g : B\xrightarrow[\text{на}]{}A$ так:
               $g(y) = \left\{\begin{matrix}
                 h(y)\text{, если} y \in B_{1} \\
                 a_{0}\text{, если}y \in B\backslash B_{1}
                \end{matrix}\right.$
         \item $b \Rightarrow a$ \\
               Пусть $g : B\xrightarrow[\text{на}]{ }A$.\\
               Построим $f : B\rightarrow A$.\\
               Рассмотрим $x\in A$\\
               Множество $\{y\in B |\ g(y) = x\}$ непусто.\\
               Выберем в качестве $f(x)$ некоторый элемент из этого множества. Проверим, что $f$ инъективна. Пусть $f(x_{1}) = f(x_{2})$\\
               Тогда $g(f(x_{1})) = g(f(x_{2}))$ , а по построению $g(f(x_{i})) = x_{i}$ при $i = 1,2.$
        \end{enumerate}
       \end{proof}
 \item Лемма о сохранении мощностей, теорема о мощности объединения (без доказательства).
       \begin{lemma*}[Лемма о сохранении мощностей]\mbox{}\\
        \begin{enumerate}
         \item Если $\left | A \right | = \left | A_{1} \right |$ и $\left | B \right | = \left | B_{1} \right |$, то $\left | A\times B \right | = \left | A_{1}\times B_{1} \right |$\\
         \item Если при этом $A\cap B = A_{1}\cap B_{1}=\emptyset $, то $\left | A \cup B \right | =\left | A_{1} \cup B_{1} \right |  $
        \end{enumerate}
       \end{lemma*}
       \begin{proof}\mbox{}\\
        \begin{enumerate}
         \item Пусть даны биекции \\$f : A\xrightarrow[\text{на}]{1-1}A_{1}$ и $g : B\xrightarrow[\text{на}]{1-1}B_{1}$.\\
                Построим $h : A \times B\xrightarrow[\text{на}]{1-1}A_{1} \times B_{1}$ так:
               $h_{1}(<x;y>)=<f(x),g(y)>$.
                Легко проверить, что $h_{1}$ - нужная биекция.
         \item Построим $h_{2} : A\cup B \xrightarrow[\text{на}]{1-1}A_{1}\cup B_{1}$ так:
               $h_{2}(x) =
                \left\{
                \begin{matrix}
                 f(x)\text{, если} x \in A \\
                 g(x)\text{, если} x \in B \\
                \end{matrix}
                \right.$
               Условие $A \cap B = \emptyset$ гарантирует, что определение корректно.
               Вновь нетрудно доказать, что $h_{2}$ - биекция. Проверим в качестве примера, что $h_{2}$ инъективна. Пусть $h_{2}(x)=h_{2}(y)$. Если $x,y \in A$, то получаем $f(x)=f(y)$ и $x=y$. Если $x,y \in B$, рассуждения аналогичны. Если же $x \in A, y \in B$  (или наоборот), то $h_{2}(x) \in A_{1}$ и $h_{2}(y) \in B_{1}$, что невозможно в силу $A_{1} \cap B_{2} = \emptyset$.
        \end{enumerate}
       \end{proof}
       \begin{lemma*}[о мощности объединения] Если хотя бы одно из множеств $A,B$ бесконечно, то $\left | A \cup B \right | = max\{\left | A \right |,\left | B \right |\}$.
       \end{lemma*}
 \item Континуум-гипотеза, теорема Гёделя-Коэна (без доказательства), обобщенная континуумгипотеза.
       \begin{hypo}[Континуум-гипотеза] Не существует множества $A$ такого, что\\
        $\left | \mathbb{N} \right | < \left | A \right | < \left | \mathbb{R} \right |$
       \end{hypo}
       \begin{theorem*}[Теорема Гёделя-Коэна]
        Если теория множеств ZFC непротиворечива, то континуум-гипотезу нельзя ни доказать, ни опровергнуть в рамках ZFC.
       \end{theorem*}
       \begin{hypo}[Обобщенная континуумгипотеза]
        Если множество $B$ - бесконечно, то не существует множества $A$ такого, что\\ $\left | B \right | < \left | A \right | < \left | P(B) \right |$
       \end{hypo}
 \item Ординалы, лемма об элементах ординала
       \begin{definition*}[Ординал]
        Ординалом называется транзитивное множество все элементы которого сравнимы относительно включения.
       \end{definition*}
       \begin{definition*}[Транзитивное множество]
        Множество $\alpha$ называется транзитивным, если из $x \in \alpha $ и $y \in x $ следует, что  $x \in \alpha $.
       \end{definition*}
       \begin{lemma*}[Лемма об элементах ординала]
        Если $\alpha$ - ординал и $\beta \in \alpha$, то $\beta$ - ординал.
       \end{lemma*}
       \begin{proof}
        Пусть $x,y\in \beta$. Тогда $x,y\in \alpha$. Следовательно, $x$ и $y$ равны или сравнимы относительно $\in$. Докажем, что $\beta$ транзитивно. Пусть $y \in x \in \beta$. Тогда $x \in \alpha$ и $y \in \alpha$. Возможны три случая:
        \begin{enumerate}
         \item $\beta \in y$ Тогда получаем, что $\beta \in y \in x \in \beta$ - противоречие.
         \item $\beta = y$ Получаем, что $\beta \in x \in \beta$ - противоречие.
         \item $y \in \beta$.
               Следовательно, $\beta$ - ординал.
        \end{enumerate}
       \end{proof}
\end{enumerate}
\end{document}
