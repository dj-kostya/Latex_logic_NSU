\documentclass{article}
\usepackage[utf8]{inputenc}
\usepackage[english, russian]{babel}
\usepackage{indentfirst}
\usepackage{amsfonts}
\begin{document}
  \begin{enumerate}
    \item Множество: способы задания, операции над множествами
    \\ Не существует явного определения множества.
    \\ Пусть A некоторое мн-во, тогда существует 2 способа задания мн-ва
          \begin{enumerate}
          \item A = \{1,2,3,4,5\} - явное задание эл-тов мн-ва \\
          \item Пусть $\Phi(x)$- некоторое условие, тогда \\A = $\{x \ | \ \Phi(x) \}$ - Задание множествами с помощью некоторого условия $\Phi(x)$
          \end{enumerate}
      Пусть A, B- некоторые множества \\
      Некоторые обозначения:
      \begin{itemize}
          \item A - подмножетсво B \\
          $A \subseteq B = \{x \ | x\in{A} \Rightarrow x\in{B} \}$
          \item A - собстевнное подмножетсво B \\
          $A \subset B$, если $A \subseteq B$ и $A\ne{B} $
          \item $\emptyset$ - множество, не содержащее эл-тов ("Пустое множество")
          \item Множество всех подмножетсв множества A
          \\ $P(A) = \{ \ C\ |\ C \subseteq{A} \ \} $

      \end{itemize}
      Операции над множествами:
      \begin{itemize}
        \item Объединение множеств:
        \\ $A\cup{B} = \{ x \ | \ x \in{A} \lor x\in{B}\}$
        \item Пересечение множеств:
        \\ $A\cap{B} = \{ x \ | \ x \in{A} \land x\in{B}\}$
        \item Разность множеств:
        \\ $A \ \backslash\  B = \{x\ | \ x\in{A} \land x\notin{B} \}$
      \end{itemize}
  \end{enumerate}
\end{document}
