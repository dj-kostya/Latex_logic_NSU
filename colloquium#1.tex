\documentclass{article}
\usepackage[utf8]{inputenc}
\usepackage[english, russian]{babel}
\usepackage{indentfirst}
\usepackage{amsfonts}
\usepackage{amsthm}
\usepackage{svg}
\newtheorem{theorem}{Теорема}
\newtheorem*{theorem*}{Теорема}
\newtheorem{lemma}[theorem]{Лемма}

\theoremstyle{definition}
\newtheorem{definition}{Определение}
\newtheorem*{definition*}{Определение}

\newtheorem*{name}{Обозначение}

\newtheorem*{exmp}{Пример}


\title{ Коллоквиум №1 (20.11.2019) }
\author{GROUPS №19137,№19144}
\date{2019\\}
\begin{document}
  \maketitle
  \begin{enumerate}
    \item Множество: способы задания, операции над множествами
    \\ Не существует явного определения множества.
    \\ Пусть A некоторое мн-во, тогда существует 2 способа задания мн-ва
          \begin{enumerate}
          \item A = \{1,2,3,4,5\} - явное задание эл-тов мн-ва \\
          \item Пусть $\Phi(x)$- некоторое условие, тогда \\A = $\{x \ | \ \Phi(x) \}$ - Задание множествами с помощью некоторого условия $\Phi(x)$
          \end{enumerate}
      Пусть A, B- некоторые множества \\
      \begin{name}[Подмножетсво]
        A - подмножетсво B, если
        \mbox{$A \subseteq B = \{x \ | x\in{A} \Rightarrow x\in{B} \}$}
      \end{name}
      \begin{name}[Собстевнное подмножетсво]
        A - собстевнное подмножетсво B, если
        \mbox{$A \subset B$, если $A \subseteq B$ и $A\ne{B} $}
      \end{name}
      \begin{name}[Пустое множество]
        $\emptyset$ - множество, не содержащее эл-тов ("Пустое множество")
      \end{name}
      \begin{name}[Множество всех подмножетсв множества A]
        \mbox{$P(A) = \{ \ C\ |\ C \subseteq{A} \ \} $}
      \end{name}
      \begin{name}[Универсум]
          Универсум (условное множество все множеств) $U$
      \end{name}
      Операции над множествами:
      \begin{itemize}
        \item Объединение множеств:
        \\ $A\cup{B} = \{ x \ | \ x \in{A} \lor x\in{B}\}$
        \item Пересечение множеств:
        \\ $A\cap{B} = \{ x \ | \ x \in{A} \land x\in{B}\}$
        \item Разность множеств:
        \\ $A \ \backslash\  B = \{x\ | \ x\in{A} \land x\notin{B} \}$
        \item Дополнение множества:
        \\ $\neg{A} = \{\ x\ |\ x \in{\ U} \land{x \notin{A}}\} $
        \item Симметрическая разность множеств:
        \\ $A \ \Delta \ B = (A \  \backslash \ B) \cup{(B \ \backslash{\ A})} = (A\ \cup{\ B })\ \backslash \ (B\ \cup{\ A } ) $

      \end{itemize}\newpage
      Пусть S - семейство множеств:
      \begin{itemize}
        \item Объединение семейства множеств\\
        $\bigcup{S} = \{\ x\ |\ \exists{A_i} \in{\ S \ }:x\ \in{\ A_i \ }\}$
        \item Пересечение семейства множеств\\
        $\bigcap{S} = \{\ x\ |\ \forall{A_i} \in{\ S\ }: x\ \in{\ A_i \ } \}$
      \end{itemize}
    \item Упорядоченный набор (кортеж), предложение о равенстве n-ок, декартово произведение, декартова степень.
    \begin{definition*}[Упорядоченный набор (кортеж)]
      Упорядоченный набор (кортеж) длинны n определяется по индукции
      \begin{math}\\
        <> = \emptyset\\
        <a> = a\\
        <a,b> = \{\{a\},\{a,b\}\}
        \\
        ...
        \\
        <a_1, a_2, ...,a_{n-1} ,a_n> = <<a_1, a_2, ...,a_{n-1})>, a_n>
      \end{math}
    \end{definition*}
    \begin{definition*}[пара]
        Набор $<a,b>$ длинны 2 называют \textit{парой}
    \end{definition*}
    \begin{theorem*}[Предложение о равенстве n-ок]
      Если \mbox{$<a_1,...,a_n> = <b_1, ..., b_n>$}, то $a_1 = b_1, ..., a_n = b_n$
    \end{theorem*}
    \begin{proof}
        ...
    \end{proof}
    \begin{definition*}[Декартово произведение]
      Пусть даны множества $A_1,...,A_n$, тогда их декартовым произведением называют
      \newline\mbox{$A_1 \times A_2 \times ... \times A_n = \{<a_1,...,a_n>|\  \forall{i}\in{\{1,...,n\}}\ a_i\in{A_i} \}$}
    \end{definition*}
    \begin{definition*}[Декартова степень]
      В случае, если \mbox{$A_1=A_2=...=A_n$}, тогда $A_1 \times A_2 \times ... \times A_n$ называют декартовой степенью и обозначают, как $A^n=A_1 \times A_2 \times ... \times A_n$
    \end{definition*}
    
  \end{enumerate}
\end{document}
